%----------------------------%
%----- 1_Einleitung.tex -----%
%----------------------------%
%----------------------------%
%
%\lipsum[2-4]
Ein elektrisches Netz im stationären Zustand kann mit komplexen Wechselstromrechnungen einfach und effizient analysiert
werden. Im Gegensatz dazu befinden sich die Zustandsvariablen Strom und Spannung im Allgemeinen während eines Ausgleichsvorgangs nicht periodisch und muss mit Differentialgleichungen (DGL) beschrieben werden. Die Lösung ist mathematisch anspruchsvoll und viel komplexer.

In diesem Experiment soll der Ausgleichsvorgang beim Anschließen oder
Abschalten der Spannungsquelle an einer RC-Reihenschaltung untersucht werden. Zu diesem Zweck sollten im ersten Schritt die theoretisch erwarteten Verläufe der Zustandsvariablen ermittelt werden.

Im zweiten Schritt soll die Schaltung aufgebaut und die Ausgleichsvorgänge aufgezeichnet werden. Im dritten Schritt soll die Schaltung in LTspice aufgebaut und simuliert werden, um dealisierte Abläufe der Ausgleichsvorgänge zu erhalten. Zusätzlich soll im vierten Schritt die Schaltung um ein weiteres RC-Element erweitert und erneut simuliert werden, um den Einfluss zusätzlicher RC-Elemente untersuchen zu können.

Abschließend sind die theoretischen, gemessenen und die aus der Simulation erhaltenen Ergebnisse zu beschreiben, zu vergleichen und zu
interpretieren. Die Inhomogenität dieser Schaltung bei Anschaltungsvorgangs wird verursacht wenn wir das folgendes betrachten:

$R C u_C'(t)+u_C(t)=U_E$ mit Anfangbedingung $u_C(0)=0$ ist $U_E$
\\
die Lösung dafür ist \[u_c(t)=U_E( 1-e^{-\frac t{RC}} )\]

Wenn es Homogene wäre 
$u_C(t)=0$ da $U_E=0$ wäre

Also die Inhomogenität verursacht dass die Spannung langsam (umgekehrt exponential) wächst von $0$ bis $U_E$ wenn $t\rightarrow\infty$
%
%
% Bitte denkt daran, eure Autorenschaft namentlich zu kennzeichnen! Das gilt für jeden (Unter-)Abschnitt, den ihr bearbeitet habt.
%
\begin{flushright}
  \textit{\autorA}
\end{flushright}
%
%
%